% Copyright 2012 David W. Hogg (NYU).
% All rights reserved.

\documentclass[12pt]{article}

\newcommand{\etal}{\textit{et al.}}
\newcommand{\ZEBRA}{\textsl{ZEBRA}}
\newcommand{\given}{\,|\,}
\newcommand{\dd}{\mathrm d}
\newcommand{\new}{\mathrm new}

\begin{document}

\noindent\textsl{This document is a draft dated 2012-04-02.}

\paragraph{To Soumagnac from Hogg:\\ Replacing the \ZEBRA\ data-driven prior PDF:}
In an email conversation from 2011 October, we looked at equation~(6)
in the \ZEBRA\ paper (Feldmann \etal, 2006, \textit{MNRAS},
\textbf{372}, 565), in which they construct a prior PDF from the
observed data.  The equation is very suspicious, although I don't
doubt that the use of it improves performance over other heurisitic
choices of prior PDF (like a ``flat prior'' or the like).  The issue
is \emph{not} that it is impossible to take a prior PDF from the data;
indeed one \emph{should} take one's prior PDF from the data!  The
prior PDF should represent the information coming from prior data.
The issue is that the correct way to take a prior PDF from the data
involves probabilistic inference, not heuristics.

I conjecture that \emph{if the \ZEBRA\ code is modified to infer its
  prior hierarchically, performance will improve even further.}
Details about how to perform this inference follow.

Converting from the notation in the \ZEBRA\ paper to more standard
probabilistic notation, the idea is that there is a discrete set of
possible redshifts $z$ and galaxy templates $t$.  The \ZEBRA\ code can
construct, for an object $i$, given observations $f_i$ (a vector or
list or blob of photometric measurements) a likelihood $p(f_i \given
z, t, I)$, which is a probability (per unit data) of getting data
$f_i$ given redshift and template pair $(z, t)$, and other prior
information $I$, which includes the all-important noise model and
uncertainty variances ($\sigma_{fi}^2$).  The code can also construct
a posterior PDF according to Bayes theorem
\begin{eqnarray}\displaystyle
P(z, t \given f_i, I) &=&
  \frac{1}{Z_i}\,p(f_i \given z, t, I)\,P(z, t)
\\
Z_i &\equiv& \sum_z\sum_t p(f_i \given z, t, I)\,P(z, t)
\quad ,
\end{eqnarray}
where $Z_i$ is a normalization factor and $P(z, t)$ is a
prior PDF for the redshift and template.  The prior and posterior PDFs
get capital $P$ function symbols because they represent discrete
probabilities over a finite number of redshifts and templates.  This
is all fine.

The \ZEBRA\ paper constructs a better-than-flat prior PDF by iterating
the following assignment to convergence:
\begin{eqnarray}\displaystyle
P_{\new}(z, t) &\leftarrow&
  P(z, t)\,\frac{1}{N}
  \,\sum_{i=1}^N \frac{1}{Z_i}\,p(f_i \given z, t, I)
\quad ,
\end{eqnarray}
where the sum is over the $N$ galaxies $i$ for which there are
observations.  Actually the sum is limited to only those galaxies with
good redshift determinations (which is one of the red flags about the
method).  This prior PDF update is iterated to convergence, along with
a bit of smoothing in the $(z, t)$ space.  The \ZEBRA\ method works
better with this iteratively constructed prior PDF than it does with
any ``flat'' prior PDF, presumably because this prior puts weight on
those redshifts and templates where the data have high likelihood;
that is, the prior PDF prefers redshifts and templates that are likely
in other data.  That's good!

However, there are two (related) problems with this approach.  The
first is that it involves summing (or averaging) probability
distribution functions.  That is rarely a good idea.  The method also
has serious dimensional problems if you take the limit that $z$ and
$t$ go over to a continuous space.  The second problem is that the
method has no fundamental justification in probability theory.  What
would that look like?

One way to think about the problem is to introduce
\emph{hyperparameters}, or parameters that control the prior PDF.  The
dumbest possible idea for hyper\-parameters is to introduce a list
$\beta$ of discrete probabilities, one for every $(z, t)$ pair.
That's not a good idea, because it can make for very non-smooth prior
PDFs.  So it is better to think of something like this: For each
template $t$ have some hyper\-parameters $\beta_t$ that set the
probability of having template $t$, and, given that template, a prior
PDF for the redshift \emph{given} that template.  In this model, the
prior PDF looks like
\begin{eqnarray}\displaystyle
P(z, t \given \beta) = P(t \given \beta)\,P(z \given t, \beta_t)
\quad,
\end{eqnarray}
where, implicitly, the hyper\-parameters $\beta_t$ are all lumped
together into one huge blob $\beta$ of hyper\-parameters.

With this new hyper\-parameterized prior PDF, the posterior PDF becomes
\begin{eqnarray}\displaystyle
P(z, t \given f_i, \beta, I) &=&
  \frac{1}{Z_i(\beta)}\,p(f_i \given z, t, I)\,P(z, t \given \beta)
\\
Z_i(\beta) &\equiv& \sum_z\sum_t p(f_i \given z, t, I)\,P(z, t \given \beta)
\quad .
\end{eqnarray}
Posterior PDFs aren't the only things you can construct with a prior
PDF; you can also construct marginalized likelihoods.  For example,
for galaxy $i$, the normalization factor $Z_i(\beta)$ is really a
marginalized probability of getting the data $f_i$ given the model and
the hyper\-parameters:
\begin{eqnarray}\displaystyle
P(f_i \given \beta, I) &=&
  \sum_z\sum_t p(f_i \given z, t, I)\,P(z, t \given \beta)
\quad ,
\end{eqnarray}
where I have done exactly what I did to get $Z_i(\beta)$ but just
interpreted it probabilistically as a marginalization.  If I think
about the \emph{full data set} $D$ as being the full list of all the
$f_i$, and if each galaxy is an independent observation (has an
independent noise draw), then there is a \emph{full-data marginalized
  likelihood}
\begin{eqnarray}\displaystyle
P(D \given \beta, I) &=&
  \prod_{i=1}^N P(f_i \given \beta, I)
\quad ;
\end{eqnarray}
this is a product not a sum.  In the absence of a prior PDF on the
hyper\-parameters---a \emph{hyperprior} if you will---the best choice of
the hyper\-parameters $\beta$ is the set that maximizes this full-data
marginalized likelihood.

It is also possible, of course, to apply hyper\-priors and either
maximize or sample from the marginalized posterior PDF $p(\beta \given
D, I)$ for the hyper\-parameters.  Indeed, one way to put in
smoothness on the redshift distribution is to make the $\beta$ very
free, put the smoothness preference into a hyperprior.  But that's
advanced!

I conjecture that a prior obtained from the data by optimizing the
full-data marginalized likelihood will outperform---by any prediction
metric---any prior obtained by any other methods, provided that it is
given similar freedom.  In fact, I think there are theorems that
more-or-less prove this.

\end{document}
